\documentclass[slidestop]{beamer} 
\usepackage[utf8]{inputenc}
\usepackage{pgf}
\usepackage{algorithmic}
\usepackage{wrapfig}
\usepackage{array}
\usepackage{subfigure}
\usepackage{xmpmulti}

\usetheme{JuanLesPins}
\usecolortheme{beaver}
\setbeamertemplate{navigation symbols}{}
\setbeamertemplate{footline}[frame number]
\title[batman-adv: distributed handling of non-mesh clients in a layer 2 mesh]{batman-adv: distributed handling of non-mesh clients in a layer 2 mesh} 
\author{Marek Lindner \& Antonio Quartulli} 
\date{May 14th, 2014\\WirelessBattleMeshv7 - Leipzig}
\institute[]{B.A.T.M.A.N.-Advanced\\www.open-mesh.org}

\newcommand{\cfbox}[2]{
	\colorlet{currentcolor}{.}
	{\color{#1}
	\fbox{\color{currentcolor}#2}}
}


\begin{document}

\begin{frame}
	\titlepage
\end{frame}

\begin{frame}[c]
	\frametitle{What we are not going to talk about today}
	\begin{itemize}
		\item backward compatibility (TVLV, compat number, etc)
		\item VLAN-ization of non-mesh client handling
		\item fragmentation v2.0 (we fragment everything!)
		\item extended AP isolation
		\item multicast improvements
		\item inter-connecting batman clouds
	\end{itemize}
\end{frame}

\section{Introduction}
\begin{frame}[c]
	\frametitle{Network wide multi-interface optimization}
	chachacha cool!
	\begin{figure}
		\centering
		\includegraphics[scale=0.4]{multi-if.jpg}
	\end{figure}
\end{frame}

\section{Routing protocol}
\begin{frame}[c]
	\frametitle{B.A.T.M.A.N. IV - the present}
	\begin{itemize}
		\item packet loss based
		\item one packet (OGM) - two functions
			\begin{itemize}
				\item Neighbour discovery/Link quality
					measurement
				\item Routing information (and more..) diffusion
			\end{itemize}
		\item not easy to tune
	\end{itemize}
	\begin{figure}
		\centering
	\end{figure}
\end{frame}

\begin{frame}[c]
	\frametitle{B.A.T.M.A.N. V - the future}
	\begin{itemize}
		\item ''throughput`` based
		\item two packets - two functions
			\begin{itemize}
				\item ELP: Neighbour discovery/Link metric
					exchange
				\item OGM:Routing information (and more..) diffusion
			\end{itemize}
		\item ELP and OGM intervals can be tuned independently
	\end{itemize}
	\begin{figure}
		\centering
	\end{figure}
\end{frame}

\begin{frame}[c]
	\frametitle{How does the routing work?}
	picture and explanation on how the metric is propagated. (Simon had one
	in his notes two years ago)
\end{frame}

\begin{frame}[c]
	\frametitle{Bandwidth: how do we get it? 1/2}
	explain the wifi case and the interaction with cfg/mac80211 (WIP)
\end{frame}

\begin{frame}[c]
	\frametitle{Bandwidth: how do we get it? 2/2}
	other cases:
	\begin{itemize}
		\item Ethernet connections (common scenario)\\
			read negotiated speed from the driver (not the best
			solution)
		\pause
		\item ADSL lines, VPN links, various tunnels\dots (less
			common)\\
			can't directly read it. Needs for an explicit
			measurement
	\end{itemize}
	\pause
	Introduction of a throughput meter in the batman-adv module (started
	GSoC 2012)
\end{frame}

\begin{frame}[c]
	\frametitle{Bandwidth meter}
	how it work..blabla like TCP, very simple explanation.
\end{frame}

\begin{frame}[c]
	\frametitle{Current status - how to enable it}
\end{frame}

\begin{frame}[c]
	\frametitle{Current status}
	work in progress
\end{frame}

\section{Conclusions}
\begin{frame}
	Nice idea, needs a lot of testing.

	\textbf{Wiki:}\\
	http://www.open-mesh.org\\
	\textbf{IRC:}\\
	#batman @ Freenode\\
	\textbf{Mailing list}
\end{frame}

\begin{frame}[c]
	\begin{center}
	\Large{\textbf{Thank you for your attention\\[1cm]
	Questions?}}
	\end{center}
\end{frame}

\end{document}
